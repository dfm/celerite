% Copyright 2015-2016 Dan Foreman-Mackey and the co-authors listed below.

\documentclass[manuscript, letterpaper]{aastex6}

\pdfoutput=1

%%% This file is generated by the Makefile.
\newcommand{\githash}{108ca20}\newcommand{\gitdate}{2016-09-06}\newcommand{\gitauthor}{Eric Agol}
\usepackage{microtype}

\usepackage{url}
\usepackage{amssymb,amsmath}
\usepackage{natbib}
\usepackage{multirow}
\bibliographystyle{aasjournal}

% ----------------------------------- %
% start of AASTeX mods by DWH and DFM %
% ----------------------------------- %

\setlength{\voffset}{0in}
\setlength{\hoffset}{0in}
\setlength{\textwidth}{6in}
\setlength{\textheight}{9in}
\setlength{\headheight}{0ex}
\setlength{\headsep}{\baselinestretch\baselineskip} % this is 2 lines in ``manuscript''
\setlength{\footnotesep}{0in}
\setlength{\topmargin}{-\headsep}
\setlength{\oddsidemargin}{0.25in}
\setlength{\evensidemargin}{0.25in}

\linespread{0.54} % close to 10/13 spacing in ``manuscript''
\setlength{\parindent}{0.54\baselineskip}
\hypersetup{colorlinks = false}
\makeatletter % you know you are living your life wrong when you need to do this
\long\def\frontmatter@title@above{
\vspace*{-\headsep}\vspace*{\headheight}
\noindent\footnotesize
{\noindent\footnotesize\textsc{\@journalinfo}}\par
{\noindent\scriptsize Preprint typeset using \LaTeX\ style AASTeX6 with modifications
}\par\vspace*{-\baselineskip}\vspace*{0.625in}
}%
\makeatother

% Section spacing:
\makeatletter
\let\origsection\section
\renewcommand\section{\@ifstar{\starsection}{\nostarsection}}
\newcommand\nostarsection[1]{\sectionprelude\origsection{#1}}
\newcommand\starsection[1]{\sectionprelude\origsection*{#1}}
\newcommand\sectionprelude{\vspace{1em}}
\let\origsubsection\subsection
\renewcommand\subsection{\@ifstar{\starsubsection}{\nostarsubsection}}
\newcommand\nostarsubsection[1]{\subsectionprelude\origsubsection{#1}}
\newcommand\starsubsection[1]{\subsectionprelude\origsubsection*{#1}}
\newcommand\subsectionprelude{\vspace{1em}}
\makeatother

\widowpenalty=10000
\clubpenalty=10000

\sloppy\sloppypar

% ------------------ %
% end of AASTeX mods %
% ------------------ %

% Projects:
\newcommand{\project}[1]{\textsl{#1}}
\newcommand{\kepler}{\project{Kepler}}

\newcommand{\foreign}[1]{\emph{#1}}
\newcommand{\etal}{\foreign{et\,al.}}
\newcommand{\etc}{\foreign{etc.}}

\newcommand{\figureref}[1]{\ref{fig:#1}}
\newcommand{\Figure}[1]{Figure~\figureref{#1}}
\newcommand{\figurelabel}[1]{\label{fig:#1}}

\newcommand{\Table}[1]{Table~\ref{tab:#1}}
\newcommand{\tablelabel}[1]{\label{tab:#1}}

\renewcommand{\eqref}[1]{\ref{eq:#1}}
\newcommand{\Eq}[1]{Equation~(\eqref{#1})}
\newcommand{\eq}[1]{\Eq{#1}}
\newcommand{\eqalt}[1]{Equation~\eqref{#1}}
\newcommand{\eqlabel}[1]{\label{eq:#1}}

\newcommand{\sectionname}{Section}
\newcommand{\sectref}[1]{\ref{sect:#1}}
\newcommand{\Sect}[1]{\sectionname~\sectref{#1}}
\newcommand{\sect}[1]{\Sect{#1}}
\newcommand{\sectalt}[1]{\sectref{#1}}
\newcommand{\App}[1]{Appendix~\sectref{#1}}
\newcommand{\app}[1]{\App{#1}}
\newcommand{\sectlabel}[1]{\label{sect:#1}}

\newcommand{\T}{\ensuremath{\mathrm{T}}}
\newcommand{\dd}{\ensuremath{\,\mathrm{d}}}
\newcommand{\unit}[1]{{\ensuremath{\,\mathrm{#1}}}}
\newcommand{\bvec}[1]{{\ensuremath{\boldsymbol{#1}}}}

% TO DOS
\newcommand{\todo}[3]{{\color{#2}\emph{#1}: #3}}
\newcommand{\dfmtodo}[1]{\todo{DFM}{red}{#1}}
\newcommand{\agoltodo}[1]{\todo{Agol}{blue}{#1}}


% \shorttitle{}
% \shortauthors{}
% \submitted{Submitted to \textit{The Astrophysical Journal}}

\begin{document}

\title{%
Fast and scalable Gaussian process modeling of stellar variability
\vspace{-3\baselineskip}  % OMG AASTEX6 IS SO BROKEN
}

\newcounter{affilcounter}
% \altaffiltext{1}{}

\setcounter{affilcounter}{1}
\edef \uw {\arabic{affilcounter}}\stepcounter{affilcounter}
\altaffiltext{\uw}       {Astronomy Department, University of Washington,
                          Seattle, WA, 98195, USA}

\edef \sagan {\arabic{affilcounter}}\stepcounter{affilcounter}
\altaffiltext{\sagan}{Sagan Fellow}

\author{%
    Daniel~Foreman-Mackey\altaffilmark{\uw,\sagan} and
    Eric~Agol\altaffilmark{\uw}
}



\begin{abstract}

We derive a fast, $O(N)$, Gaussian Process approach that involves Lorentzian
kernels.  This basis set can approximate commonly used kernels, and is a good
description of stellar variability.

\end{abstract}

\keywords{%
% methods: data analysis
% ---
% methods: statistical
% ---
% catalogs
% ---
% planetary systems
% ---
% stars: statistics
}

\section{Introduction}

\section{Outline}

\begin{enumerate}
\item Non-parametric modeling of time series with correlated noise using Gaussian processes
\item Assumptions:\\
  a). Stationary noise (although this can be broken).\\
  b). Gaussian
\item Problem: can't handle large datasets - e.g. entire Kepler long-cadence; Spitzer light
   curves; Solar curves; short-cadence.
\item Solution(s):\\
  a). HODLR \citep{Ambikasaran2013,Ambikasaran2016}\\
  b). CARMA (Lorentzian power spectra, but with limitations) \citep{Kelly2014}\\
  c). Tri-diagonal (Press \& Rybicki) \citep{1995PhRvL..74.1060R}
\item Problem: HODLR is tricky to use; CARMA requires certain relations between
  coefficients of Lorentzians (right?); P\&R only works for up to 2 components + white noise
\item Solution: Ambikasaran \citep{Ambikasaran2015} shows how sum of exponential kernels can be solved in order
   $O(N*p^2)$, where p is number of Lorentzians.
\item Generalized Ambikasarn for complex coefficients (Hermitian).
\item Performance:\\
  a). Show $O(N*p^2)$ scaling\\
  b). Show comparison to HODLR.\\
  c). Show comparison to wavelet (ala Carter \& Winn).\\
  d). Compare with CARMA Kallman filter(?).
\item Various methods/formulae:\\
  a). Likelihood computation.\\
  b). Generating GP with particular covariance [yet to solve]. [ ]\\
  c). Inferring particular components.\\
  d). Computing derivatives of GPs (as used in RV methods).\\
  e). Forecasting/interpolating.\\
  f). Inferring confidence intervals.\\
  g). Derivative of likelihood function with respect to kernel parameters.\\
  h). Non-stationary GP (can the coefficients vary with time?).\\
  i). Multi-band (each element of extended matrix becomes a matrix).

\item Generalization to non-stationary noise.
\item Example application: TYC 3559\\
  a). Entire long-cadence 17-quarter Kepler dataset (~65k data points).\\
  b). Analysis of solar data.

\item Possible future applications:\\
  a). Inference of asteroseismic variability - ala Andrew Gordon Wilson \citep{2013arXiv1302.4245W}, but
      with Lorentzians, not Gaussians.\\
  b). Multi-waveband GPs.\\
  c). RV interpolation.\\
  d). Non-parametric phase functions.\\
  e). Better calibration of the flicker log(g) relation.\\
  f). Measurement of rotational periods - gyrochronology.\\
  g). Doppler shifts in oscillating EBs.

\item Appendix:
  a). Description/summary of mathematics.\\
  b). Description of code.
\end{enumerate}

\subsection{Derivative of likelihood with respect to Kernel parameters}

The derivative of the likelihood can be computed in $O(N)$ operations as follows.

The derivative of the log likelihood with respect to model parameters ${\bf x}$ is given by:
\begin{equation}
    \frac{\partial \ln{\cal L}}{\partial x_i} = (\ln{\cal L})^\prime = -\frac{1}{2}\sum_i \frac{K_{ii}^\prime}{K_{ii}}
    - \frac{1}{2} {\bf y}^T {\bf K}^{-1} {\bf K}^\prime {\bf K}^{-1} {\bf y},
\end{equation}
where $^\prime$ means the derivative wrt $x_i$.
The kernel may be written as:
\begin{equation}
    K_{ij}({\bf \alpha},{\bf \beta}) = \left\{
    \begin{array}{cc}
       w_i + \sum_{l=1}^p \alpha_l & j = i \\
       \sum_{l=1}^p \alpha_l \exp{(-\beta_l |t_i-t_j|)}  & j \ne i
    \end{array}
    \right.
\end{equation}
Since this equation is the sum of $p$ semi-separable matrices, then applying the product rule to compute the derivative will give a single semi-separable matrix for the derivative with respect to each kernel parameter.

Now, letting ${\bf x} = (w,{\bf \alpha},{\bf \beta})$, then the derivative may be written as:
\begin{eqnarray}
    \frac{\partial K_{ii}}{\partial w} &=& \delta_{ij}\\
    \frac{\partial K_{ij}}{\partial \alpha_k} &=&
     \left\{
    \begin{array}{cc}
       1 & j = i \\
        \exp{(-\beta_k |t_i-t_j|)}  & j \ne i
    \end{array}
    \right.\\
    \frac{\partial K_{ij}}{\partial \beta_k} &=&
     \left\{
    \begin{array}{cc}
       0  & j = i \\
        -\alpha_k|t_i-t_j|\exp{(-\beta_k |t_i-t_j|)}  & j \ne i
    \end{array}
    \right.
\end{eqnarray}
The first term in the derivative of the likelihood can be computed in $O(N)$
evaluations, while the second term can be computed in three steps:
\begin{itemize}
\item First, solve ${\bf z_1} = {\bf K}^{-1}{\bf y}$ with the GRP solver.
\item Next, create an e
\end{itemize}
Writing this term out:

\section{Summary}

\vspace{1.5em}
All of the code used in this project is available from
\url{https://github.com/dfm/GenRP} under the MIT open-source software
license.
This code (plus some dependencies) can be run to re-generate all of the
figures and results in this paper; this version of the paper was generated
with git commit \texttt{\githash} (\gitdate).


\acknowledgments
It is a pleasure to thank
...
for helpful contributions to the ideas and code presented here.

EA acknowledges support from NASA grants NNX13AF20G, NNX13AF62G, and
NASA Astrobiology Institute’s Virtual Planetary Laboratory, supported
by NASA under cooperative agreement NNH05ZDA001C.

This research made use of the NASA \project{Astrophysics Data System} and the
NASA Exoplanet Archive.
The Exoplanet Archive is operated by the California Institute of Technology,
under contract with NASA under the Exoplanet Exploration Program.

This paper includes data collected by the \kepler\ mission. Funding for the
\kepler\ mission is provided by the NASA Science Mission directorate.
We are grateful to the entire \kepler\ team, past and present.

These data were obtained from the Mikulski Archive for Space Telescopes
(MAST).
STScI is operated by the Association of Universities for Research in
Astronomy, Inc., under NASA contract NAS5-26555.
Support for MAST is provided by the NASA Office of Space Science via grant
NNX13AC07G and by other grants and contracts.

\facility{Kepler}
\software{%
    % \project{ceres} \citep{Agarwal:2016},
    % \project{corner.py} \citep{Foreman-Mackey:2016},
    % \project{emcee} \citep{Foreman-Mackey:2013},
    % \project{george} \citep{Ambikasaran:2016},
	% \project{matplotlib} \citep{Hunter:2007},
	% \project{numpy} \citep{Van-Der-Walt:2011},
	% \project{scipy} \citep{Jones:2001}.
}

\appendix

\bibliography{genrp}

\end{document}
